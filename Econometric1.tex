\documentclass[12pt]{article}
\usepackage{geometry} 
\geometry{head = 2cm, bottom = 1cm, left = 2cm, right = 2cm}
\usepackage{amsmath,amsthm,amssymb,scrextend}
\usepackage{fancyhdr}
\setlength{\headheight}{14.5pt}
\addtolength{\topmargin}{-2.5pt}
\pagestyle{fancy}

\newcommand{\cont}{\subseteq}
\usepackage{tikz}
\usepackage{pgfplots}
\usepackage{amsmath}
\usepackage[mathscr]{euscript}
\let\euscr\mathscr\let\mathscr\relax% just so we can load this and rsfs
\usepackage[scr]{rsfso}
\usepackage{amsthm}
\usepackage{amssymb}
\usepackage{multicol}
\usepackage[colorlinks=true, pdfstartview=FitV, linkcolor=blue,
citecolor=blue, urlcolor=blue]{hyperref}
\usepackage{graphicx}
\usepackage{float}
\usepackage{listings}
\usepackage{color}
\usepackage{xcolor}
\definecolor{codegreen}{rgb}{0,0.6,0}
\definecolor{codegray}{rgb}{0.5,0.5,0.5}
\definecolor{codepurple}{rgb}{0.58,0,0.82}
\lstset{
			backgroundcolor=\color{white},
            frame = single,
            basicstyle=\footnotesize\ttfamily,
            columns=fullflexible,
			breaklines = true,
			tabsize=4,
            numbers=left,
			commentstyle=\color{codegreen},
            otherkeywords = {0,1,2,3,4,5,6,7,8,9},
            deletekeywords={<-},
            keywordstyle=\color{blue},
			numberstyle=\tiny\color{black},
			stringstyle=\color{codepurple},
	    }
\DeclareMathOperator{\arcsec}{arcsec}
\DeclareMathOperator{\arccot}{arccot}
\DeclareMathOperator{\arccsc}{arccsc}
\newcommand{\ddx}{\frac{d}{dx}}
\newcommand{\dfdx}{\frac{df}{dx}}
\newcommand{\ddxp}[1]{\frac{d}{dx}\left(#1 \right)}
\newcommand{\dydx}{\frac{dy}{dx}}
\newcommand{\intx}[1]{\int#1 \, dx}
\newcommand{\intt}[1]{\int#1 \, dt}
\newcommand{\defint}[3]{\int_{#1}^{#2} #3 \, dx}
\newcommand{\imp}{\Rightarrow}
\newcommand{\un}{\cup}
\newcommand{\inter}{\cap}
\newcommand{\ps}{\mathscr{P}}
\newcommand{\set}[1]{\left\{ #1 \right\}}
\newtheorem*{sol}{Solution}
\newtheorem*{claim}{Claim}
\newtheorem{problem}{Problem}
\pgfplotsset{compat=1.17}
\begin{document}
 
% Don't change the above session

\lhead{Econometric}
\chead{Computer Homework 1}
\rhead{\today}

% \maketitle
\begin{large}
    \hspace*{-5.8mm}Group 9
    \end{large}\\\\
    \hspace*{4mm} 107205008 \emph{Yu-Chen, Den}\\
    \hspace*{4mm} 108208014 \emph{Fu-Chien, Hsiao}\\
    \hspace*{4mm} 109102057 \emph{Liang-Yu, Hsu}

\section*{Ch11-C4}
\subsubsection*{Find the first difference model}
\begin{lstlisting}[language = R]
install.package("wooldridge")
library(wooldridge)
data("phillips")
head(phillips)
tail(phillips)
fdm <- lm(data = phillips, cnif ~ cunem)
summary(fdm)
\end{lstlisting}
\begin{figure}[H]
    \centering
    \includegraphics*[width = 0.6\textwidth]{fdm.png}
    \caption{summary of fdm}
\end{figure}
\number1. $\beta_1$ = -0.83281 and the P-value is 0.00583 $<$ 0.01\\
\number2. The fdm model fits better, because the $\hat R^2$ is bigger.
\newpage

\section*{Ch11-C8}
\subsubsection*{Estimate the AR (1) model and predict $unem$ of 2004}
\begin{lstlisting}[language = R]
AR1= arima(phillips$unem,order=c(1,0,0))
AR1
5.5631+0.7493*phillips$unem[length(phillips$unem)]
plot.ts(phillips$unem)
\end{lstlisting}
The predict value of $unem$ = 10.0589, and the true $unem$ = 5.5

\subsubsection*{Add a lag of inflation to the AR (1) model}
\begin{lstlisting}[language = R]
AR2= arima(phillips$unem,order=c(1,0,0), xreg=phillips$inf_1)
AR2
(1-pnorm(abs(AR2$coef)/sqrt(diag(AR2$var.coef))))*2
\end{lstlisting}
\begin{figure}[H]
    \centering
    \includegraphics*[width = 0.7\textwidth]{AR2.png}
    \caption{summary of AR1}
\end{figure}
the t-ratio of $inf_{t-1}$
\[\frac{0.1277}{0.0581} = 2.1979 > t_{0.025}(30)\simeq 1.96\]
We can say that $inf_{t-1}$ is significant.\\

\subsubsection*{Predict the $unem$ of 2004}
\begin{lstlisting}[language = R]
(5.1810+0.7090*phillips$unem[length(phillips$unem)]+
    0.1277*phillips$inf_1[length(phillips$inf_1)])
\end{lstlisting}
The predict value of $unem$ in 2004 = 9.63932
\newpage

\section*{Ch12-C10}
\subsubsection*{Estimate the static Phillips curve equation}
\begin{lstlisting}[language = R]
library(wooldridge)
str(phillips)
ph1 <- phillips
model1 <- lm(data = ph1, inf~unem)
summary(model1)
\end{lstlisting}
\begin{figure}[H]
    \centering
    \includegraphics*[width = 0.6\textwidth]{M1.png}
    \caption{summary of model1}
\end{figure}
    
\subsubsection*{See if serial correlation exists}
\begin{lstlisting}[language = R]
u_round<- round(residuals(model1),digits = 2)
c1 <- c(0,u_round)
c2 <- c(u_round,0)
m1 <- matrix(0, nrow = 57, ncol = 2)
m1[,1] <- c1
m1[,2] <- c2
colnames(m1) <- c("ut_1","ut")
data1 <- data.frame(m1)
data_c <- data1[-c(1,57),]
model2 <- lm(ut ~ ut_1, data = data_c)
summary(model2)
\end{lstlisting}
\begin{figure}[H]
    \centering
    \includegraphics*[width = 0.6\textwidth]{M2.png}
    \caption{summary of model2}
\end{figure}
The coefficient of $u_{t-1}$ = 0.5723, and the t-ratio = 5.281, so there's strong evidence of serial correlation.
\newline

\subsubsection*{Find out if there is much difference when the later years are added}
\begin{lstlisting}[language = R]
install.packages("prais")
library(prais)
p1 <- prais_winsten(inf ~ unem, data = ph1, index = "year")
summary(p1)
\end{lstlisting}
\begin{figure}[H]
    \centering
    \includegraphics*[width = 0.5\textwidth]{M3.png}
    \caption{summary of static Phillips curve model by iterative Prais-Winsten}
\end{figure}
New coefficient of $unem$ is -0.714, and it has only small difference between the origin -0.716
\end{document}
