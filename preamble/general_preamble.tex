%% ========== Geometry Settings ==========
\usepackage{fancyhdr}
\usepackage{geometry}
\geometry{top = 2cm, bottom = 2cm, right = 2cm, left = 2cm}
\usepackage{appendix}

%% ========== Figure Settings ==========
\usepackage{graphicx}
\usepackage{float}
\usepackage{subcaption}

%% ========== Table Settings ==========
\usepackage{multirow}
\usepackage{booktabs}
\usepackage{longtable}
\usepackage{dcolumn}
\usepackage{ctable}
%% ========== Math Settings ==========
\usepackage{amsmath, mathtools, amssymb, empheq, amsthm, cancel, mleftright} % 
\DeclareMathOperator{\erf}{erf}
\DeclareMathOperator{\arcsec}{arcsec}
\DeclareMathOperator{\arccot}{arccot}
\DeclareMathOperator{\arccsc}{arccsc}
\DeclareMathOperator{\supp}{supp}
\DeclareMathOperator{\diag}{diag}
\DeclareMathOperator*{\argmax}{arg\,max}
\DeclareMathOperator*{\argmin}{arg\,min}
\renewcommand\qedsymbol{$\blacksquare$}
\allowdisplaybreaks
\newcommand{\ddx}{\frac{d}{dx}}
\newcommand{\dfdx}{\frac{df}{dx}}
\newcommand{\ddxp}[1]{\frac{d}{dx}\left(#1 \right)}
\newcommand{\dydx}{\frac{dy}{dx}}
\let\ds\displaystyle
\newcommand{\intx}[1]{\int #1 \, dx}
\newcommand{\intt}[1]{\int #1 \, dt}
\newcommand{\imp}{\Rightarrow}
\newcommand{\un}{\cup}
\newcommand{\inter}{\cap}
\newcommand{\ps}{\mathscr{P}}
\newcommand{\set}[1]{\left\{ #1 \right\}}

%% ========== TIKZ Settings ==========
\usepackage{tikz}
\usepackage{standalone}
\tikzset{>=latex} % for LaTeX arrow head
\usepackage{pgfplots} % for the axis environment
\usepackage{xcolor}
\usepackage[outline]{contour} % halo around text
\contourlength{1.2pt}
\usetikzlibrary{positioning,calc}
\usetikzlibrary{backgrounds}% required for 'inner frame sep'
\pgfplotsset{compat = 1.18}

%% ========== Reference Settings ==========
\usepackage{hyperref} % Set some hyperref style
\hypersetup{
    colorlinks = true,
    linkcolor = blue,
    filecolor = magenta,      
    urlcolor = sinopia,
}
\urlstyle{same}

%% ========== Algorithm ============
\usepackage[ruled]{algorithm2e}

%% ========== Font Settings ==========
\sffamily % sans serif font